%%%%%%%%%%%%%%%%%%%%%%%%%%%%%%%%%%%%%%%%%%
% Enrique Acosta, 2022
%%%%%%%%%%%%%%%%%%%%%%%%%%%%%%%%%%%%%%%%%%
% To make gif
% ----------------------------------------
% *  Have gnuplot installed (brew install gnuplot)
% 
% *  Compile this with shellscape: 
%    pdflatex -shell-escape gnuplotTest.tex
% 
% *  Make gif with convert:
%    convert -density 300 -delay 5 -loop 0 -alpha remove gnuplotTest.pdf gnuplotTest.gif
%%%%%%%%%%%%%%%%%%%%%%%%%%%%%%%%%%%%%%%%%%


\documentclass[border=2pt, tikz]{standalone}
\usepackage{pgfplots}

\pgfplotsset{compat=newest}

\begin{document}

    \def\a{1}
    \def\b{5.8}

\foreach \a in {-4, -3.8, ..., 4}{
    \begin{tikzpicture}
    \begin{axis}[xmin=-3, xmax=3, ymin=-8, ymax=8]
    
    \addplot +[
        no markers,
        raw gnuplot,
        thick,
        empty line = jump % not strictly necessary, as this is the default behaviour in the development version of PGFPlots
     ] gnuplot {
        f(x,y)=x*(x^2+y^2)-(\a*y+\b*x);
        set xrange [-3:3];
        set yrange [-8:8];
        set cntrparam levels incremental 0.001, 0.001, 0.001;
        set cont base;
        set view 0,0;
        unset surface;
        set isosamples 100,100;
        splot f(x,y);
    };

    % Equation with white around it
    \node[right, blue, fill=white] at (-2,6) {$x(x^2+y^2)=(ay+\b x)$};

    % Print the value of a
    \pgfkeys{/pgf/number format/.cd,fixed};
    \node[right] at (-2.7,-6) {$a=\pgfmathprintnumber{\a}$};

    \end{axis}
    \end{tikzpicture}
}

\end{document}