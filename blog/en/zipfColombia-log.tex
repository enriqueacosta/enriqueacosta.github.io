% Enrique Acosta
% 2025
\documentclass[border=2pt, tikz, convert={command=\unexpanded{magick -density 600 \infile\space "\jobname.png"}}]{standalone}
\usepackage{pgfplots}
\usepackage{pgfplotstable}
% \pgfplotsset{compat=1.17}

\pagecolor{white} % Set the background color to white

% Data table (city names with spaces are in braces!)
\pgfplotstableread{
rank  pop       city
1     7937898   Bogotá
2     2634570   Medellín
3     2285099   Cali
4     1342818   Barranquilla
5     1065881   Cartagena
6     828947    Soacha
7     815891    Cúcuta
8     686339    Soledad
9     623881    Bucaramanga
10    593273    Villavicencio
11    575225    Valledupar
12    570329    Bello
13    566650    {Santa Marta}
14    546003    Ibagué
15    531424    Montería
16    482824    Pereira
17    459262    Manizales
18    415937    Pasto
19    388229    Neiva
20    359888    Palmira
}\citydata

\begin{document}
\begin{tikzpicture}
\begin{axis}[
    width=12cm,
    height=8cm,
    xlabel={Rank},
    ylabel={Population},
    title={Colombia City Populations: Log-Log Scale},
    xmin=0.9, xmax=20.5,
    ymin=300000, ymax=10000000,
    xmode=log, ymode=log,
    log ticks with fixed point,
    xtick={1,...,20},
    xticklabels={1,...,10,,,,,15,,,,,20},
    ytick ={300000,600000,1200000,2400000,4800000,9600000},
    % ytick num = {4},
    legend pos=north east,
    grid=both,
    tick label style={font=\small},
    ylabel style={yshift=1.5em},
    ymajorgrids=true,
    xmajorgrids=true,
]

% Power law fit
\addplot [
    domain=1:20,
    samples=100,
    very thick,
    dashed,
    cyan,
] {5544989 * x^(-0.92)};


% Scatter plot: city data
\addplot[
    only marks,
    mark=*,
    black,
    mark size=2.5pt,
] table [x=rank, y=pop] {\citydata};


% Fit equation
\node[rectangle, draw, fill=white, text=cyan!70!black, anchor=south west] at (axis cs: 4, 1800000) {$P(r)=\displaystyle\frac{5544989}{r^{0.92}}$};

\end{axis}


% link at the bottom
\node[anchor=south east, font=\tiny, text=gray!80!black, inner sep=2pt] at (current bounding box.south east) {enriqueacosta.github.io 2025};



\end{tikzpicture}
\end{document}

